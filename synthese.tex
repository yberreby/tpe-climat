% Preamble
% ---
\documentclass[12pt,a4paper]{article}

% Packages
% ---
%\usepackage{amsmath} % Advanced math typesetting
\usepackage[utf8]{inputenc} % Unicode support (Umlauts etc.)w
\usepackage[french]{babel} % Change hyphenation rules
\usepackage[T1]{fontenc} % for farenheit
\usepackage{textcomp}
\usepackage{gensymb}
\usepackage{hyperref}
\usepackage{graphicx}
\graphicspath{ {images/} }
%\usepackage{listings} % Source code formatting and highlighting

\begin{document}

% Custom title page
\begin{titlepage}
	\centering
	
	{\scshape\Large TPE Mathématiques et SVT\par}
	\vspace{0.2cm}	
	Thème : l'aléatoire, l'insolite, le prévisible\par
	Axe de recherche : comprendre le présent, penser le futur\par
	\vspace{1.5cm}

	{\scshape\LARGE La modélisation informatique du climat \par}
	\vspace{1cm}
	{\huge\bfseries Comment peut-on modéliser le rayonnement solaire, et traduire cette modélisation mathématique en programme informatique ?\par}

	\vspace{1cm}
	{\Large\itshape Yohaï Eliel Berreby\par}
	
	\vfill
	
	
	{\Large\bfseries Établissement : CNED\par
	Série : S }

	\vfill

	{\large \today\par}
\end{titlepage}


\section{Le choix du sujet}

Avant de m’emparer du sujet, j’ai voulu explorer la façon dont la musique et les
mathématiques dialoguent et identifier les principaux domaines d’étude. J’ai
référencé :

\subsection{foo}
\end{document}
