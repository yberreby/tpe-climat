% Preamble
% ---
\documentclass[12pt,a4paper]{article}

% Packages
% ---
%\usepackage{amsmath} % Advanced math typesetting
\usepackage[utf8]{inputenc} % Unicode support (Umlauts etc.)w
\usepackage[french]{babel} % Change hyphenation rules
\usepackage[T1]{fontenc} % for farenheit
\usepackage{textcomp}
\usepackage{gensymb}
\usepackage{hyperref}
\usepackage{graphicx}
\graphicspath{ {images/} }
%\usepackage{listings} % Source code formatting and highlighting

\begin{document}

% Custom title page
\begin{titlepage}
	\centering
	
	{\huge\bfseries  TPE Mathématiques et SVT\par}
	\vspace{0.2cm}	
	{ \Large Thème : l'aléatoire, l'insolite, le prévisible\par }
	{ \large Axe de recherche : comprendre le présent, penser le futur\par }
	\vspace{1.5cm}

	{\bfseries \scshape\Large La modélisation informatique du climat \par}
	\vspace{1cm}
	
	{\Huge\bfseries  \scshape Note de synthèse \par}
		\vspace{1cm}
		
	
	{\Large Comment peut-on modéliser l'irradiation solaire, et traduire cette modélisation mathématique en programme informatique ?\par}

	\vspace{1cm}
	{\Large\itshape Yohaï Eliel Berreby\par}
	
	\vfill
	
	
	{\Large\bfseries Établissement : CNED\par}
	\vspace{0.2cm}
	{\Large \bfseries Série : S }

	\vfill

	{\large \today\par}
\end{titlepage}

% La note de synthèse doit rappeler la démarche suivie, les raisons du choix du sujet, le bilan personnel. Elle doit permettre de comprendre la cohérence du TPE.



\section{Le choix du sujet}

%  Quels sont les éléments qui vous ont amené à choisir ce sujet ?
%  Justifier dans le cadre de ce sujet les disciplines retenues. 

Je m'intéresse à l'informatique, et plus particulièrement à la programmation, depuis maintenant plusieurs années.
Cet intérêt vient notamment du lien étroit qu'il existe entre cette discipline et les mathématiques, ainsi que des immenses possibilités qu'elle ouvre à une époque où les ordinateurs prennent une importance grandissante.
Ils sont aujourd'hui utilisés dans des domaines aussi divers que la finance, le divertissement, la médecine, ou encore la prévision météorologique.


C'est ce dernier point, la prévision météorologique, qui m'a fait m'arrêter sur le thème de \emph{"l'aléatoire, l'insolite, le prévisible"}.


En effet, l'axe de recherche \emph{"comprendre le présent, penser le futur"} de ce thème me parut une superbe opportunité d'étudier l'application des mathématiques et de l'informatique à la prévision.


La \textbf{prévision météorologique} s'imposa rapidement dans mon esprit comme 

L.
De plus, étant donné que la météorologie fait beaucoup appel aux Sciences de la Terre, j'ai trouvé que ce sujet apporterait un caractère interdisciplinaire à mon travail.

Je me suis finalement décidé à retenir le sujet de la \textbf{modélisation informatique du climat}, et j'ai commencé ma recherche de problématique.
C'est donc naturellement que j'ai commencé ma recherche de sujet dans l'espoir de pouvoir associer l'informatique à mon travail,


 
\section{Ma démarche}

% Cheminement qui a conduit à la formulation de la problématique (changement de direction, reformulation de la problématique…) 
% Moyens mis en œuvre pour répondre à la problématique (les réussites, les impasses, les limites…)


\section{Bilan personnel}

\section{Conclusion}


\end{document}
