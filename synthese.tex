% Preamble
% ---
\documentclass[12pt,a4paper]{article}

% Packages
% ---
%\usepackage{amsmath} % Advanced math typesetting
\usepackage[utf8]{inputenc} % Unicode support (Umlauts etc.)w
\usepackage[french]{babel} % Change hyphenation rules
\usepackage[T1]{fontenc} % for farenheit
\usepackage{textcomp}
\usepackage{gensymb}
\usepackage{hyperref}
\usepackage{graphicx}
\graphicspath{ {images/} }
%\usepackage{listings} % Source code formatting and highlighting

\begin{document}

% Custom title page
\begin{titlepage}
	\centering
	
	{ \Huge \bfseries Note de synthèse\par}
	\vspace{1cm}

	{\scshape\Large TPE Mathématiques et SVT\par}
	\vspace{0.2cm}	
	Thème : l'aléatoire, l'insolite, le prévisible\par
	Axe de recherche : comprendre le présent, penser le futur\par
	\vspace{1.5cm}

	{\scshape\LARGE La modélisation informatique du climat \par}
	\vspace{1cm}
	{\huge\bfseries Comment peut-on modéliser le rayonnement solaire, et traduire cette modélisation mathématique en programme informatique ?\par}

	\vspace{1cm}
	{\Large\itshape Yohaï Eliel Berreby\par}
	
	\vfill
	
	
	{\Large\bfseries Établissement : CNED\par
	Série : S }

	\vfill

	{\large \today\par}
\end{titlepage}


\section{Le choix du sujet}

L'informatique me passionne depuis de nombreuses années.
Lorsqu'il me fallut commencer la recherche d'un sujet pour mon TPE, j'ai donc naturellement tenté d'en trouver un qui me permette de tirer parti, dans mon travail, de cette passion qui a l'avantage d'être intimement liée aux mathématiques.
 

\subsection{foo}
\end{document}
