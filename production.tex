% Préambule
% ---
\documentclass[12pt]{article}

% Paquets utilisés
% ---
\usepackage{amsmath} % Advanced math typesetting
\usepackage[utf8]{inputenc} % Unicode support (Umlauts etc.)w
\usepackage[french]{babel} % Change hyphenation rules
\usepackage[T1]{fontenc} % for farenheit
\usepackage{textcomp}
\usepackage{gensymb}
\usepackage{hyperref}
\usepackage[detect-weight=true, detect-family=true]{siunitx}
\usepackage{xcolor}
\usepackage{placeins}
\usepackage{float}

\usepackage{graphicx}
\graphicspath{{../resources/images/}}

\usepackage{minted}
\renewcommand{\theFancyVerbLine}{\sffamily \textcolor[rgb]{0.0,0.0,0.0}{\oldstylenums{\arabic{FancyVerbLine}}}}


\begin{document}

%%
% Page de couverture sur-mesure
%%

\begin{titlepage}
	\centering
	
	{\huge\bfseries  TPE Mathématiques et SVT\par}
	\vspace{0.2cm}	
	{ \Large Thème : l'aléatoire, l'insolite, le prévisible\par }
	{ \large Axe de recherche : comprendre le présent, penser le futur\par }
	\vspace{1.5cm}

	{\bfseries \scshape\Large La modélisation informatique du climat \par}
	\vspace{1cm}
	{\Large Comment peut-on modéliser l'irradiation solaire, et traduire cette modélisation mathématique en programme informatique ?\par}

	\vspace{1cm}
	{\Large\itshape Yohaï Eliel Berreby\par}
	
	\vfill
	
	
	{\Large\bfseries Établissement : CNED\par}
	\vspace{0.2cm}
	{\Large \bfseries Série : S }

	\vfill

	{\large \today\par}
\end{titlepage}


\clearpage
\tableofcontents{}
\clearpage

%%
% Début du document
%%

\section{Introduction} 

Les prévisions météorologiques font aujourd'hui partie intégrante de la vie quotidienne.
Qui n'a jamais eu besoin d'allumer sa télévision ou de consulter son téléphone pour savoir quel temps il ferait avant de prendre un avion, de partir en vacances ou tout simplement d'aller se promener en forêt ?
Cependant, bien qu'elles soient communes et faciles d'accès, les prévisions météorologiques présentent des enjeux importants.
Ces enjeux sont \emph{économiques}, avec par exemple la météo agricole, un outil précieux pour des millions d'agriculteurs à travers le monde, mais aussi \emph{humains}, car prévoir une crue ou une tempête ne serait-ce que quelques heures à l'avance peut permettre de sauver des vies en évacuant les populations en danger.

Les prévisions météorologiques sont aujourd'hui littéralement à \emph{portée de main} avec l'avènement des applications mobiles, et pourtant les constructions sur lesquelles elles reposent sont d'une grande complexité.
Il s'agit d'outils à la croisée des mathématiques, de la physique et de l'informatique : les \textbf{modèles}.

\subsection{Les modèles en sciences du climat}

\subsubsection{Modéliser pour prévoir}
Les modèles sont des représentations de phénomènes rendues suffisamment simples pour qu'on puisse les appréhender et les manipuler, tout en restant aussi proches que possible de la réalité.
Dans le cadre de la météorologie, c'est-à-dire l'étude du temps à court terme, ils sont surtout utiles pour réaliser des \textbf{prévisions}.
Les modèles \emph{climatiques}, eux, permettent une étude du temps à plus long terme, en simulant les conditions climatiques des centaines, des milliers ou des millions d'années dans le futur ou dans le passé.

La modélisation du temps, qu'elle soit météorologique ou climatique, est également aujourd'hui indissociable de l'informatique, d'où le choix du sujet "la modélisation informatique du climat".
En effet, le volume de données dépassant rapidement de très loin les capacités humaines, on a recours à des ordinateurs pour réaliser les calculs nécessaires, car ils peuvent le faire à des vitesses très élevées.

Malgré cette puissance de calcul grandissante et l'utilisation de modèles mathématiques toujours plus précis, les prévisions météorologiques restent des estimations.
Nous pouvons donc prévoir le temps, mais non le prédire.


\subsubsection{La modélisation de l'irradiation solaire}

Le rôle primordial des modèles dans un domaine aussi important que la météorologie m'a conduit à vouloir explorer leur fonctionnement en créant un modèle à petite à échelle.
J'ai décidé de m'intéresser à l'irradiation solaire car son intensité, appelée éclairement énergétique solaire, est relativement simple à calculer avec une marge d'erreur raisonnable.
En effet, elle est plus stable, plus prévisible que le vent ou les intempéries, par exemple.
Bien que mon modèle se restreigne à cette seule variable, les calculs qu'il implique sont suffisamment complexes pour justifier la création d'un programme informatique les réalisant automatiquement.

Pour mener à bien mon projet, j'ai d'abord tenté de comprendre le phénomène de l'irradiation solaire.
Je me suis ensuite attaché à la modéliser par le calcul à l'aide de ressources trouvées sur Internet, afin d'aboutir à une méthode de calcul de son intensité au niveau du sol.
Finalement, j'ai intégré ces calculs dans un programme informatique, et je l'ai testé en le faisant fonctionner et en comparant les résultats obtenus à ceux attendus.


\section{Les phénomènes gouvernant l'irradiation solaire}

On appelle \emph{éclairement énergétique solaire} la puissance reçue du Soleil par unité de surface.
Cette valeur est mesurée en \SI{}{\watt\per\square\meter}.

\subsection{L'atténuation du rayonnement solaire}
\label{sec:attenuation-rayonnement}
L'intensité du rayonnement solaire frappant la Terre variant peu, l'éclairement énergétique \emph{au sol} dépend principalement du niveau d'atténuation des rayons du Soleil lors de leur passage à travers l'atmosphère.
En effet, alors que dans l'espace les rayons du Soleil peuvent parcourir de grandes distances en perdant peu d'énergie, ils en perdent beaucoup par diffusion et absorption sitôt qu'ils pénètrent l'atmosphère.
Plus la distance parcourue au travers de l'atmosphère est grande, plus ils perdent d'énergie.
Cette distance est d'autant plus importante que les rayons solaires suivent une trajectoire inclinée, c'est-à-dire que leur angle d'incidence est grand (voir figure ci-dessous).

\centerline{\includegraphics[width=\textwidth]{angle-of-incidence.jpg}}


\subsection{L'angle zénithal}
\label{sec:zenith-angle}
L'angle d'incidence des rayons solaires est appelé \emph{angle zénithal}.
Il varie en fonction de trois paramètres : \emph{l'angle horaire}, \emph{la déclinaison}, et la \emph{latitude}.

\textbf{L'angle horaire} est la mesure du déplacement apparent du Soleil dans le ciel. 
Il change avec l'heure : c'est pour cela qu'il fait moins chaud le soir qu'à midi, car comme le Soleil s'approche de l'horizon, ses rayons suivent une trajectoire oblique et sont fortement atténués par l'atmosphère.

La \textbf{déclinaison}, elle, est l'angle que forment le plan de l'équateur et les rayons du Soleil.
Elle varie tout au long de l'année,  ce qui fait que la surface de la Terre est éclairée différemment par le Soleil tout au long de l'année : c'est le cycle des saisons (voir figure~\ref{fig:seasons-year}).

La déclinaison et la \textbf{latitude} doivent être prises en compte car les saisons ne sont pas les mêmes partout sur le globe.
Ainsi, par exemple, il fera plus froid en janvier qu'en juillet dans l'hémisphère Nord, mais ce sera l'inverse dans l'hémisphère Sud car l'été y dure de décembre à février, et l'hiver de juin à août.

 \begin{figure}[H]
	\centerline{\includegraphics[width=\textwidth]{seasons-2.jpeg}}
	\caption{La déclinaison vaut $0\degree$ aux équinoxes de printemps et d'automne, $23,45\degree$ au solstice d'été et $-23,45\degree$ au solstice d'hiver.}
	\label{fig:seasons-year}
\end{figure}
	

\subsection{Angle horaire, heure solaire et heure légale}

On a vu plus tôt que l'angle zénithal dépendait, entre autres, de l'angle horaire.
Ce dernier est exprimable indépendamment en degrés ou en heures, et est appelé \emph{heure solaire} dans le second cas.
Le midi solaire est ainsi défini comme ``l'instant où le Soleil atteint son point de culmination, en un endroit donné de la Terre'' \cite{temps_solaire_wiki} : à ce même instant, l'angle horaire vaut 90\degree.

 \begin{figure}[H]
	\centerline{\includegraphics[width=\textwidth]{midi-solaire.jpg}}
	\caption{Le midi solaire.}
	\label{fig:solar-noon}
\end{figure}

On pourrait penser que l'heure solaire ou l'angle horaire peuvent être calculés simplement à partir de l'heure \emph{légale}, ou heure locale, mais ce n'est pas le cas.
Il existe un décalage variable entre ces deux valeurs, car l'heure solaire évalue de façon irrégulière, contrairement à l'heure légale.



\subsubsection{L'équation du temps}
Ce décalage entre l'heure légale et l'heure solaire existe car l'orbite de la Terre autour du Soleil est irrégulière.
En effet, elle n'est pas parfaitement circulaire, mais elliptique (on peut le voir sur la figure~\ref{fig:seasons-year}), et l'attraction qu'exerce la Lune sur la Terre perturbe également son orbite.

Ces irrégularités sont représentées par un paramètre appelé \emph{équation du temps}\footnote{\textit{``Remarque sur le mot ``équation'' : en astronomie ancienne, le terme ``équation'' désignait une correction ajoutée algébriquement à une valeur moyenne pour obtenir une valeur vraie.
C'est une telle acception qui a survécu dans l'expression ``équation du temps'' ''} \cite{equation_temps_wiki}.} \cite{equation_temps_wiki}.
Il s'agit de la variation de l'heure solaire par rapport à l'heure solaire moyenne.
Sa courbe est similaire d'année en année.

 \begin{figure}[H]
	\centerline{\includegraphics[width=\textwidth]{eot-graph.png}}
	\caption{
		La courbe de l'équation du temps en 1993.
	}
	\label{fig:eot-graph}
\end{figure}

\subsubsection{La variation de l'heure solaire au sein d'un fuseau horaire}
L'équation du temps n'est cependant pas suffisante pour calculer l'heure solaire ; il faut aussi prendre en compte la \emph{longitude} du point où l'on se place.
En effet, au sein d'un fuseau horaire donné, l'heure légale est partout la même, mais ce n'est pas le cas de l'heure solaire.
Le Soleil ne sera ainsi pas aussi haut dans le ciel à un instant donné si l'on se place dans deux villes situées aux deux extrémités du même fuseau horaire (par exemple Dallas et Chicago, toutes deux situées dans le fuseau horaire UTC-06:00).


\section{Approche de la modélisation mathématique}

Nous avons tenté d'expliquer les principaux phénomènes physiques qui influent sur l'intensité du rayonnement solaire.
Nous allons à présent essayer de les modéliser mathématiquement afin de calculer l'éclairement énergétique solaire en un point.
%, en fonction de ces paramètres (position sur Terre, date et heure).
%Pour cela, nous utiliserons une approche descendante, pour calculer les paramètres intermédiaires nécessaires

\subsection{Éclairement énergétique et masse d'air}

Nous avons vu dans la partie~\ref{sec:attenuation-rayonnement} que plus l'épaisseur d'atmosphère traversée par les rayons solaires est grande, plus ils perdent d'énergie.
On utilise pour représenter cette épaisseur un paramètre appelé \emph{masse d'air} (\textit{air mass} en anglais) \cite{air_mass_wiki}.
Celui-ci permet de calculer directement l'éclairement énergétique solaire au sol.
Plus cette valeur est élevée, plus les rayons du Soleil sont atténués.

On note $I_d$ l'éclairement énergétique solaire au niveau du sol, exprimé en \SI{}{\watt\per\square\meter} (watts par mètre carré), $AM$ (pour \emph{air mass}) la masse d'air, et $\theta$ l'angle zénithal (voir section~\ref{sec:zenith-angle}).

On a alors \cite{pos_air_mass} :

\[
	\scalebox{1.3}{
		$I_d = 1,353 \cdot 0,7^{{AM}^{0,678}}$
	}
\]

\[
	\scalebox{1.3}{
		$AM = \frac{1}{\cos \theta}$
	}
\]

\begin{figure}[H]
	\centerline{\includegraphics[width=\textwidth - 1cm]{air-mass-schema.jpg}}
	\caption{La masse d'air $AM$ à un instant et en un point donné est définie comme la longueur du chemin que les rayons du Soleil prennent au travers de l'atmosphère ($b$), divisée par la longueur minimum de ce chemin ($a$). $b$, ainsi que la masse d'air, est minimal quand le Soleil est exactement à la verticale du point étudié.}
	\label{fig:air-mass}
\end{figure}


\subsection{Angle zénithal, d'élévation, de déclinaison}
On calcule l'angle zénithal $\theta$ à partir d'une valeur intermédiaire : \emph{l'angle d'élévation}, qu'on note $\alpha$.
Il s'agit de l'angle entre l'horizontale et le Soleil, là où l'angle zénithal est l'angle entre la \emph{verticale} et le Soleil.

L'angle zénithal $\theta$ et l'angle d'élévation $\alpha$ étant complémentaires, on a :
\[
	\theta = 90\degree - \alpha
\]

 \begin{figure}[H]
	\centerline{\includegraphics[width=\textwidth]{elevation-and-zenith-angle-ink.jpg}}
	\caption{Angle zénithal et angle d'élévation.}
	\label{fig:elevation-and-zenith-angle}
\end{figure}

$\alpha$ est plus complexe à calculer, car il est fonction de la latitude du point étudié, de \emph{l'angle de déclinaison} et de \emph{l'angle horaire}.
On a ainsi, pour $\phi$ la latitude du point étudié  \cite{pos_elevation_angle} :

\[
	\alpha = \sin^{-1} \left(
		\sin \delta \sin \phi +
		\cos \delta \cos \phi \cos \left( \text{angle horaire} \right)
	\right)
\]

L'angle de déclinaison $\delta$ est \emph{``l'angle entre la droite joignant les centres du Soleil et de la Terre et le plan de l'équateur terrestre''} \cite{mouvement_terre}. Il dépend de la position de la Terre autour du Soleil, et donc du \emph{jour de l'année} (allant de 1 à 365).

\[
	\delta = \sin^{-1} \left(
		\sin \left( 23,45\degree \right)
		\sin \left(
			\frac{360}{365}
			\left(
				\text{jour de l'année} - 81
			\right)
		\right)
	\right)
\]



\subsection{Angle horaire et temps solaire}

Pour calculer l'heure solaire à partir de l'heure locale, il faut calculer une valeur appelée facteur de correction de l'heure, qui est la différence entre l'heure solaire et l'heure locale.
On note $\Delta T_{GMT}$ le décalage horaire, en heures, par rapport à l'heure GMT.
Ainsi, par exemple, $\Delta T_{GMT}$ vaudra $1$ pour Paris, car son fuseau horaire est GMT+1.

\[
	\text{angle horaire} = 15 \degree \left(
		\text{heure solaire} - 12
	\right)
\]

\[
	\text{heure solaire} = \text{heure locale} + \frac{\text{facteur de correction de l'heure}}{60}
\]

\begin{multline*}
	\text{facteur de correction de l'heure} = 4 \left( \text{longitude} - \text{longitude du méridien local} \right) \\ + \text{équation du temps}
\end{multline*}

\begin{gather*}
	B = \frac{360}{365} \left( d - 81 \right) \\
	\text{équation du temps} = 9,87 \sin \left( 2B \right) - 7,53 \cos \left( B \right) - 1,5 \sin \left( B \right)
\end{gather*}

\[
	\text{longitude du méridien local} = 15 \degree \cdot \Delta T_{GMT}
\]


\subsection{Combinaison des paramètres}
La combinaison de tous ces paramètres permet de calculer une approximation de l'éclairement énergétique solaire en un point à partir de la position de ce point, de la date, de l'heure locale et du fuseau horaire seulement.
Le fuseau horaire doit être indiqué explicitement car on ne peut le déduire simplement à partir de la latitude, les limites des fuseaux horaires tendant à suivre les frontières des pays au lieu d'être rectilignes, comme le montre la figure~\ref{fig:timezones}.

\begin{figure}[!ht]
  \centering
      \makebox[\textwidth]{\includegraphics[width=\paperwidth]{timezones.png}}
  \caption{Les fuseaux horaires, dont les limites sont représentées par des lignes rouges sur ce schéma, suivent les contours des pays.}
  \label{fig:timezones}
\end{figure}

Cette approximation ne tient pas compte des nuages ou d'autres phénomènes climatiques pouvant altérer le parcours des rayons du Soleil, mais reste assez proche de la réalité par temps clair.

\clearpage
\begin{figure}[!htb]
	\centering
	{ \Large \textbf{Structure du modèle} \par\medskip }
	\centerline{\includegraphics[width=\paperwidth - 2cm]{graph.png}}
	\caption{Structure du modèle. Chaque bulle représente un paramètre, dépendant d'autres paramètres. Ainsi, par exemple, l'heure solaire dépend de l'heure locale et du facteur de correction de l'heure, qui dépend lui-même de la longitude du point étudié, de celle du méridien local et de l'équation du temps.}
	\label{fig:model-structure}
\end{figure}



\section{Modélisation informatique : écriture du programme}

On dispose à présent d'un petit modèle climatique.
Pour le faire fonctionner sur ordinateur, il va falloir le traduire en un algorithme qu'on intègrera dans un programme informatique.

\subsection{Bases du fonctionnement d'un programme}

Pour être exécuté, un programme doit être traduit en une série d'instructions appelée \emph{code machine}.
Ce code n'étant pas conçu pour être lu ou écrit par des êtres humains, on écrit en pratique les programmes dans des \emph{langages de programmation}, conçus pour être compréhensibles à la fois par le programmeur et par l'ordinateur.

Le texte formant un programme, écrit dans un langage de programmation, est appelé \emph{code source} du programme.
L'ordinateur traduit le code source en code machine à l'aide un programme appelé \emph{compilateur}.
Il est alors libre d'exécuter le code machine.


\subsection{Choix du langage}

Il existe de nombreux langages de programmation, chacun ayant ses propres atouts et points faibles.
Pour ce TPE, j'ai choisi d'écrire mon programme en \textbf{Rust} \footnote{\url{https://www.rust-lang.org/}}.
Il s'agit d'un langage innovant, extrêmement performant, et prévenant de nombreux bugs grâce à un système qui analyse automatiquement le code source afin de vérifier son exactitude avant la traduction en langage machine.
Ce sont ces qualités, ainsi que ma familiarité avec ce langage, qui ont motivé mon choix.


\subsection{Interface utilisateur}

J'ai décidé de munir mon programme d'une interface utilisateur simple, afin qu'il soit possible de l'utiliser sans connaissances en programmation.
Il s'agit d'une interface dite \emph{en ligne de commande}, et non d'une interface graphique traditionnelle, afin préserver la simplicité du code.
On l'utilise avec le clavier uniquement, en entrant la latitude, la longitude puis la date, l'heure et le fuseau horaire.
Ces derniers sont entrés simultanément, sous la forme \texttt{DD/MM/YYYY HH:MM:SS [fuseau horaire]}.

\begin{figure}[!htbp]
  \centering
      \makebox[\textwidth]{\includegraphics[width=15cm]{program-demonstration.png}}
  \caption{Un exemple d'utilisation de mon programme, avec pour paramètres les coordonnées GPS de la tour Eiffel (48.8582° N, 2.2945° E) et la date et l'heure d'écriture de ces lignes. Seuls les paramètres d'entrée (latitude, longitude, date et heure locale) ont été écrits au clavier ; les variables intermédiaires et le résultat ont été calculés et affichés automatiquement.}
  \label{fig:demo}
\end{figure}


\FloatBarrier
\subsection{Code source}

Le code source de mon programme est trop long\footnote{252 lignes au total, dont 176 lignes de code Rust (en excluant les commentaires et les espaces).} pour le copier dans son intégralité, mais je l'ai rendu disponible sur GitHub, un site de développement collaboratif.\footnote{\url{https://github.com/yberreby/rust-climate}}
Vous pouvez en télécharger une copie en cliquant sur "Download ZIP" dans l'interface GitHub.\footnote{\url{https://github.com/yberreby/rust-climate/archive/master.zip}}
Il est également possible de le lire directement depuis votre navigateur, en cliquant sur les noms des fichiers dans l'interface Web.
Le code est dans \texttt{src}.

En voici un extrait (fichier \texttt{src/irradiance.rs}) :

\inputminted[linenos]{rust}{rust-climate/src/irradiance.rs}



\section{Validation expérimentale : essai du programme}

Maintenant qu'on dispose d'un programme fonctionnel, il est nécessaire de le \emph{tester} pour vérifier si les résultats qu'il produit correspondent à la réalité.

\subsection{Résultats attendus}

D'après Green Rhino Energy, une société spécialisée dans l'énergie solaire \cite{green_rhino_irradiance}, l'éclairement énergétique solaire \emph{maximal} au niveau de la mer serait d'environ \SI{960}{\watt\per\square\meter}.
Cette valeur ne peut être atteinte que lorsque les rayons du Soleil arrivent perpendiculairement au plan auquel appartient le point étudié.
Ce n'est \emph{pas} le cas tous les jours au midi solaire, contrairement à ce que l'on pourrait penser.
Si c'était le cas, tous les pays seraient, chaque jour de l'année, au midi solaire, soumis à une irradiation solaire équivalente, ce qui est évidemment faux.


\subsection{Résultats obtenus}

Par définition, l'éclairement énergétique maximal est atteint lorsque la masse d'air est égale à $1$.
Cependant, on va en pratique chercher à la faire \emph{tendre} vers cette valeur, car tenter de l'égaler exactement serait inutile en raison de l'imprécision des valeurs mesurées servant de base au calcul de l'éclairement énergétique.

J'ai donc fait fonctionner mon programme avec divers paramètres jusqu'à obtenir une masse d'air tendant vers $1$, et j'obtins un éclairement énergétique maximal de \textbf{\SI{947.07}{\watt\per\square\meter}} au niveau de l'équateur, au Gabon (0\degree N, 13\degree E), à 12h20 heure locale (midi heure solaire), le 22 septembre 2016 (équinoxe d'automne).
Cette valeur est proche des \SI{960}{\watt\per\square\meter} annoncés par Green Rhino Energy : l'essai du programme est donc concluant.

\begin{figure}[!ht]
  \centering
      \makebox[\textwidth]{\includegraphics[width=15cm]{program-maximum.png}}
  \caption{Mon programme affiche un éclairement énergétique solaire maximal de \SI{947.07}{\watt\per\square\meter}, ce qui est proche des 960 à \SI{1000}{\watt\per\square\meter} attendus.}
  \label{fig:maximum}
\end{figure}


\FloatBarrier
\section{Conclusion}

Ainsi, après m'être assuré de comprendre les phénomènes physiques impliqués dans l'irradiation solaire, je les ai intégrés dans un modèle mathématique simple. J'ai ensuite écrit un programme informatique permettant de faire fonctionner ce modèle en réalisant automatiquement tous les calculs nécessaires.

L'éclairement énergétique, que mon modèle permet de calculer, est une donnée qui peut servir de base à une construction plus élaborée.
Ainsi, avec des connaissances et des moyens suffisants, on pourrait imaginer rendre ce modèle plus précis en y ajoutant, par exemple, la prise en compte des nuages, qui absorbent beaucoup d'énergie solaire, ou de la courbure de la Terre\footnote{Le calcul de la masse d'air est en effet une simplification, qui considère le sol comme un plan et non comme la surface d'une sphère, ce qui entraîne de petites imprécisions.}.
On pourrait aussi le rendre capable de prévoir la température, mais il faudrait pour cela qu'il intègre l'effet de serre, les vents, le phénomène de convection atmosphérique, et mille autres données !
On se rend bien vite compte que la complexité augmente de façon exponentielle avec le nombre de facteurs pris en considération.

Cet exemple illustre les défis que doivent relever les chercheurs en sciences du climat : si la modélisation \emph{approximative} d'une seule variable nécessite tout de même de nombreux calculs et environ 200 lignes de code, à combien plus forte raison la construction d'un modèle climatique complet, comprenant non seulement l'irradiation solaire mais aussi les vents, la mesure dans laquelle les matériaux à la surface de la Terre absorbent le rayonnement solaire, la formation de nuages, l'humidité de l'air, etc., ne mobilisera-t-elle pas d'incroyable ressources matérielles et humaines !


\clearpage

\section{Glossaire}

\textbf{angle zénithal} --- angle d'incidence des rayons solaires.
Il s'agit de l'angle que formeraient les rayons du Soleil avec une droite perpendiculaire à la surface.

\textbf{angle d'élévation} --- angle que formeraient les rayons du Soleil avec le plan de la surface.

\textbf{angle de déclinaison} --- angle entre la droite joignant les centres du Soleil et de la Terre et le plan de l'équateur terrestre.

\textbf{éclairement énergétique solaire} --- puissance transmise à un corps par le rayonnement solaire par unité de surface, exprimée en \SI{}{\watt\per\square\meter} dans le système international.

\textbf{irradiation solaire} --- exposition d'un corps à un flux de rayonnements en provenance du soleil.

\textbf{climatologie} --- science du climat. Elle porte sur de longues périodes de temps.

\textbf{code machine} --- code binaire d'un programme, généré par un \textbf{compilateur}. Il est exécutable directement par le processeur de l'ordinateur.

\textbf{code source} --- texte écrit par le programmeur représentant les instructions d'un programme.

\textbf{compilateur} --- programme qui traduit le \textbf{code source} d'un programme en \textbf{code machine}.

\textbf{modèle} --- simulation, représentation simplifiée d'un système complexe au moyen d'équations et de relations.

\textbf{équation du temps} --- différence entre l'heure solaire et l'heure solaire moyenne.


\clearpage
\begin{thebibliography}{99}

\bibitem{properties_of_sunlight} 
	PVEducation.org,
	\emph{Properties of Sunlight}\\
	\url{http://www.pveducation.org/pvcdrom/properties-of-sunlight/}

\bibitem{pos_air_mass}
	PVEducation.org,
	\emph{Air Mass}\\
	\url{http://www.pveducation.org/pvcdrom/properties-of-sunlight/air-mass}

\bibitem{pos_solar_time}
	PVEducation.org,
	\emph{Solar Time}\\
	\url{http://www.pveducation.org/pvcdrom/properties-of-sunlight/solar-time}

\bibitem{pos_elevation_angle}
	PVEducation.org,
	\emph{Elevation Angle}\\
	\url{http://www.pveducation.org/pvcdrom/properties-of-sunlight/elevation-angle}

\bibitem{pos_declination_angle}
	PVEducation.org,
	\emph{Declination Angle}\\
	\url{http://www.pveducation.org/pvcdrom/properties-of-sunlight/declination-angle}

\bibitem{pos_calculation_of_solar_insolation}
	PVEducation.org,
	\emph{Calculation of Solar Insolation}\\
	\url{http://www.pveducation.org/pvcdrom/properties-of-sunlight/calculation-of-solar-insolation}

\bibitem{mouvement_terre}
	Université du Maine,
	\emph{Mouvement de la Terre}\\
	\url{http://ressources.univ-lemans.fr/AccesLibre/UM/Pedago/physique/02/divers/mouveter.html}

\bibitem{air_mass_wiki}
	Wikipédia,
	\emph{Air Mass (solar energy)}\\
	\url{https://en.wikipedia.org/wiki/Air_mass_(solar_energy)}

\bibitem{equation_temps_wiki}
	Wikipédia,
	\emph{Équation du temps}\\
	\url{https://fr.wikipedia.org/wiki/%C3%89quation_du_temps}
	
\bibitem{irradiation_solaire_wiki}
	Wikipédia,
	\emph{Irradiation solaire}\\
	\url{https://fr.wikipedia.org/wiki/Irradiation_solaire}
	
\bibitem{temps_solaire_wiki}
	Wikipédia,
	\emph{Temps solaire}\\
	\url{https://fr.wikipedia.org/wiki/Temps_solaire}
	
\bibitem{earth_irradiance_wiki}
	Wikipédia,
	\emph{Solar Irradiance - Earth}\\
	\url{https://en.wikipedia.org/wiki/Solar_irradiance#Earth}
	
\bibitem{green_rhino_irradiance}
	Green Rhino Energy,
	\emph{Annual Solar Irradiance, Intermittency and Annual Variations}\\
	\url{http://www.greenrhinoenergy.com/solar/radiation/empiricalevidence.php}

\end{thebibliography}


\end{document}
