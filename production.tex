% Préambule
% ---
\documentclass[12pt]{article}

% Paquets utilisés
% ---
\usepackage{amsmath} % Advanced math typesetting
\usepackage[utf8]{inputenc} % Unicode support (Umlauts etc.)w
\usepackage[french]{babel} % Change hyphenation rules
\usepackage[T1]{fontenc} % for farenheit
\usepackage{textcomp}
\usepackage{gensymb}
\usepackage{hyperref}
\usepackage[detect-weight=true, detect-family=true]{siunitx}
\usepackage{xcolor}
\usepackage{placeins}
\usepackage{float}

\usepackage{graphicx}
\graphicspath{{../resources/images/}}

\usepackage{minted}
\renewcommand{\theFancyVerbLine}{\sffamily \textcolor[rgb]{0.0,0.0,0.0}{\oldstylenums{\arabic{FancyVerbLine}}}}


\begin{document}

%%
% Page de couverture sur-mesure
%%

\begin{titlepage}
	\centering
	
	{\scshape\large TPE Mathématiques et SVT\par}
	\vspace{0.2cm}	
	{ \Large Thème : l'aléatoire, l'insolite, le prévisible\par }
	{ \large Axe de recherche : comprendre le présent, penser le futur\par }
	\vspace{1.5cm}

	{\scshape\LARGE La modélisation informatique du climat \par}
	\vspace{1cm}
	{\huge\bfseries Comment peut-on modéliser l'irradiation solaire, et traduire cette modélisation mathématique en programme informatique ?\par}

	\vspace{1cm}
	{\Large\itshape Yohaï Eliel Berreby\par}
	
	\vfill
	
	
	{\Large\bfseries Établissement : CNED\par}
	\vspace{0.2cm}
	{\Large \bfseries Série : S }

	\vfill

	{\large \today\par}
\end{titlepage}


\clearpage
\tableofcontents{}
\clearpage

%%
% Début du document
%%

\section{Introduction} 

\subsection{Des enjeux actuels}

Qui n'a jamais eu besoin d'allumer sa télévision ou de consulter son téléphone pour savoir quel temps il ferait avant de prendre un avion, de partir en vacances ou tout simplement d'aller se promener en forêt ?
Les prévisions météorologiques font aujourd'hui partie intégrante de la vie quotidienne.
Communes et faciles d'accès, elles présentent cependant des enjeux importants.
Ces enjeux sont \textbf{économiques}, avec par exemple la météo agricole, un outil précieux pour des millions d'agriculteurs à travers le monde, mais aussi \textbf{humains}, car prévoir une crue ou une tempête ne serait-ce que quelques heures à l'avance peut permettre de sauver des vies en évacuant les populations en danger.

Ces prévisions sont aujourd'hui littéralement à \emph{portée de main} avec l'avènement des applications mobiles, et pourtant elles reposent sur des constructions d'une grande complexité à la croisée des mathématiques, de la physique et de l'informatique : \textbf{les modèles}.

\subsection{Les modèles en climatologie}

On parle de modèles mathématiques, scientifiques, climatiques, ou encore informatiques.
Ces notions, bien que diverses, se rejoignent en un point : il s'agit de représentations de phénomènes rendues suffisamment simples pour qu'on puisse les appréhender et les manipuler, mais qui restent les plus proches possibles de la réalité, sans quoi elles seraient inutiles. % XXX: trop long
En ce sens, l'on peut dire que tout modèle est une \textbf{approximation du réel}.

Le modèle \emph{mathématique} est une construction formelle qui sert bien souvent de base au modèle informatique.
Les lois qui régissent le premier servent de base au second, qui, soumis à des contraintes physiques (vitesse de calcul, de transfert des données, capacité de stockage d'un volume de données grandissant), en est la traduction empirique et approximative.
Bien qu'on sache aujourd'hui prévoir des phénomènes \emph{précisément à court terme}, ou \emph{approximativement à long terme}, nous sommes incapables de les prévoir \emph{précisément à long terme} : nous savons quelle température il fera demain, mais pas s'il pleuvra le même jour l'année prochaine.

Il convient d'établir une distinction entre climatologie et météorologie, car là où la météorologie étudie le temps à court terme, l'étude du climat se fait nécessairement à long terme. Prévoir s'il pleuvra dans une semaine, par exemple, relève ainsi de la météorologie, alors qu'étudier l'impact qu'aura le réchauffement climatique sur notre atmosphère dans 50 ans est du domaine de la climatologie.


\subsection{Rapport au thème}

La modélisation climatique s'inscrit naturellement dans le thème choisi, à savoir \emph{"l'aléatoire, l'insolite, le prévisible"}, car, outre l'exploration du passé qu'elle permet -- un point sur lequel on reviendra plus loin --, l'un de ses principaux intérêts est la réalisation de \textbf{prévisions météorologiques}. % XXX: IL FAUT REVENIR SUR L'EXPLORATION DU PASSÉ

Elle est également aujourd'hui indissociable de l'informatique, d'où le choix du sujet \emph{"la modélisation informatique du climat"}.
En effet, le volume de données que doit traiter tout modèle climatique à partir d'un certain degré de complexité est bien trop important pour qu'un être humain, même muni d'une calculatrice, puisse réaliser les calculs nécessaires. On a donc pour cela recours à des ordinateurs, car ils sont capables de réaliser ces calculs à des vitesses dépassant de très loin les capacités humaines.


\subsection{Démarche expérimentale}

Le rôle primordial des modèles dans un domaine aussi important que la météorologie m'a conduit à vouloir explorer leur fonctionnement en créant un modèle simple de \emph{l'irradiation solaire}.

Cet exemple, d'une grande simplicité en comparaison avec les modèles qu'utilisent des organismes comme Météo France, présente cependant de multiples avantages :

\begin{itemize}
  \item il est réalisable par un élève seul dans le cadre d'un TPE,
  \item les calculs qu'il implique restent cependant suffisamment complexes pour \textbf{justifier la création d'un programme informatique} les réalisant automatiquement,
  \item l'irradiation solaire est \textbf{relativement simple à modéliser} avec une marge d'erreur raisonnable car elle est plus stable, plus prévisible que le vent ou les intempéries, par exemple.
\end{itemize}

\vspace{0.5cm}

Pour le réaliser, j'ai tenté de comprendre les phénomènes et paramètres gouvernant l'irradiation solaire.
Ensuite, je me suis attaché à modéliser cette irradiation solaire par le calcul à l'aide de ressources trouvées sur Internet.
Finalement, j'ai intégré ces calculs dans un programme informatique, et procédé à la validation expérimentale de celui-ci en le faisant fonctionner et en comparant les résultats obtenus à ceux attendus.



\clearpage
\section{Les phénomènes gouvernant l'irradiation solaire}

On appelle \emph{éclairement énergétique solaire} la puissance reçue du Soleil par unité de surface.
Cette valeur est mesurée en \SI{}{\watt\per\square\meter}.
L'intensité du rayonnement solaire frappant la Terre variant peu, l'éclairement énergétique \emph{au sol} dépend principalement du niveau d'atténuation des rayons du Soleil lors de leur passage à travers l'atmosphère.
En effet, alors que dans l'espace les rayons du Soleil peuvent parcourir de grandes distances en perdant peu d'énergie, ils en perdent beaucoup par diffusion et absorption sitôt qu'ils pénètrent l'atmosphère.
Plus la distance parcourue au travers de l'atmosphère est grande, plus ils perdent d'énergie.
Cette distance est d'autant plus importante que les rayons solaires suivent une trajectoire inclinée, c'est-à-dire que leur angle d'incidence est grand (figure~\ref{fig:angle-of-incidence}).

 \begin{figure}[H]
	\centerline{\includegraphics[width=\textwidth]{angle-of-incidence.jpg}}
	\caption{L'angle d'incidence des rayons lumineux est appelé \emph{angle zénithal}. Plus il est grand, plus les rayons solaires doivent parcourir une distance importante au travers de l'atmosphère, et plus ils sont atténués.}
	\label{fig:angle-of-incidence}
\end{figure}

\section{L'angle zénithal}
L'angle d'incidence des rayons solaires est appelé \emph{angle zénithal}.
Il varie en fonction de trois paramètres : \emph{l'angle horaire}, \emph{l'angle de déclinaison}, et la \emph{latitude}.

L'angle horaire est la mesure du déplacement apparent du Soleil dans le ciel. 
Il change avec l'heure : c'est pour cela qu'il fait moins chaud le soir qu'à midi, car comme le Soleil s'approche de l'horizon, ses rayons suivent une trajectoire oblique et sont fortement atténués par l'atmosphère.

L'angle zénithal dépend également de la période de l'année et de la latitude car la surface de la Terre est éclairée différemment par le Soleil tout au long de l'année : c'est le cycle des saisons.
Ces différences d'ensoleillement sont dues à l'inclinaison de la Terre sur son axe (voir figure~\ref{fig:seasons-year}).
La latitude doit être prise en compte, car les saisons ne sont pas les mêmes partout sur le globe.
Ainsi, par exemple, il fera plus froid en janvier qu'en juillet dans l'hémisphère Nord, mais ce sera l'inverse dans l'hémisphère Sud car l'été y dure de décembre à février, et l'hiver de juin à août.

 \begin{figure}[H]
	\centerline{\includegraphics[width=\textwidth]{seasons-2.jpeg}}
	\caption{La Terre est inclinée sur son axe, ce qui fait qu'elle se présente au Soleil sous des angles différents à chaque saison. \textit{Crédits image : \url{http://quelles-dates.fr/dates-solstices-ete-hiver-et-equinoxes/}}}
	\label{fig:seasons-year}\end{figure}
	

\subsection{Angle horaire, heure solaire et heure légale}

On a vu plus tôt que l'angle zénithal dépendait, entre autres, de l'angle horaire.
Ce dernier est exprimable indépendamment en degrés ou en heures, et est appelé \emph{heure solaire} dans le second cas.
Le midi solaire est ainsi défini comme ``l'instant où le Soleil atteint son point de culmination, en un endroit donné de la Terre''\cite{temps_solaire_wiki} : à ce même instant, l'angle horaire vaut 90\degree.

On pourrait penser que l'heure solaire ou l'angle horaire peuvent être calculés simplement à partir de l'heure \emph{légale}, ou heure locale, (i.e., celle indiquée par une montre), mais ce n'est pas le cas.
Il existe un décalage variable entre ces deux valeurs, car l'heure solaire évalue de façon irrégulière, contrairement à l'heure légale.
Cela est dû à la trajectoire elliptique que suit la Terre dans sa révolution autour du Soleil, qu'on peut observer sur la figure~\ref{fig:seasons-year}.
La vitesse de révolution de la Terre est également variable selon l'époque de l'année.
Un paramètre appelé \emph{équation du temps}\footnote{\textit{``Remarque sur le mot ``équation'' : en astronomie ancienne, le terme ``équation'' désignait une correction ajoutée algébriquement à une valeur moyenne pour obtenir une valeur vraie.
C'est une telle acception qui a survécu dans l'expression ``équation du temps'' ''} \cite{equation_temps_wiki}.} représente la variation de l'heure solaire par rapport à l'heure solaire moyenne, et permet donc de corriger les irrégularités de celles-ci \cite{equation_temps_wiki}.

L'équation du temps n'est cependant pas suffisante pour calculer l'heure solaire ; il faut aussi prendre en compte la \emph{longitude} du point où l'on se place.
En effet, au sein d'un fuseau horaire donné, l'heure légale est partout la même, mais ce n'est pas le cas de l'heure solaire.
Le Soleil ne sera ainsi pas aussi haut dans le ciel à un instant donné si l'on se place dans deux villes situées aux deux extrémités du même fuseau horaire (par exemple Dallas et Chicago, toutes deux situées dans le fuseau horaire UTC-06:00).



% longitude méridien standard local


\clearpage
\section{Modélisation mathématique}

XXX \emph{l'éclairement énergétique solaire} d'un point en fonction de sa position sur Terre, de la date et de l'heure

\subsection{Construction du modèle}

Pour ce faire, je me suis principalement appuyé sur une collection de ressources du site PVEducation.org appelée \emph{Properties of Sunlight} \cite{properties_of_sunlight}.
À l'aide de ce site, j'ai déterminé comment l'on pouvait calculer l'éclairement énergétique solaire (\textit{solar irradiance} en anglais) à partir d'autres paramètres physiques, pouvant eux-mêmes être calculés à partir d'autres paramètres, et ainsi de suite jusqu'à établir une relation entre date, heure, position sur la Terre d'un point, et éclairement énergétique solaire de ce point.
Le tout forme un modèle à petite échelle, dont voici le fonctionnement.

\subsubsection{Calcul de de la masse d'air}


% XXXX
On utilise pour la représenter un paramètre appelé \emph{masse d'air} (ou \emph{masse atmosphérique} -- \textit{air mass} en anglais) \cite{air_mass_wiki}.
Il représente le rapport de la longueur du plus court chemin possible au travers de l'atmosphère, à savoir celui que prendraient des rayons arrivant perpendiculairement au plan de la surface, sur celle du chemin emprunté par la lumière à un instant donné (voir figure~\ref{fig:air-mass}).
Plus cette valeur est élevée, plus les rayons du Soleil doivent parcourir une distance importante, et plus ils sont atténués par l'atmosphère.

\begin{figure}[H]
	\centerline{\includegraphics[width=\textwidth]{air-mass.png}}
	\caption{La masse d'air à un instant et en un point donné (\textit{air mass} en anglais) est définie par X (chemin le plus court que peuvent prendre les rayons du Soleil) divisé par Y (chemin qu'ils prennent à cet instant et en ce point).} % C'EST FAUX - 1/COS THETA - CE N'EST PAS ÇA
	\label{fig:air-mass}
\end{figure}



D'après \emph{PVEducation.org} \cite{pos_air_mass}, on a :

\[
	I_d = 1,353 \cdot 0,7^{{AM}^{0,678}}
\]

Avec $I_d$ l'éclairement énergétique solaire en \SI{}{\watt\per\square\meter}, et $AM$ la masse d'air.

La distance parcourue par les rayons du Soleil ($Y$ sur la figure~\ref{fig:air-mass}) est égale à $\cos \theta$ ($\theta$ est l'angle zénithal, que je décrirai plus loin).
La masse d'air est donc égale à :

\[\frac{1 }{\cos \theta}\]

\subsubsection{Angle zénithal, d'élévation, de déclinaison}
$\theta$ est \emph{l'angle zénithal}. L'angle zénithal et l'angle d'élévation ($\alpha$) étant complémentaires, on a :
\[
	\delta = 90\degree - \alpha
\]

%\begin{figure}[!ht]
%	\centerline{\includegraphics[width=7cm]{elevation-and-zenith-angle.png}}
%	\caption{Angle zénithal et angle d'élévation}
%	\label{fig:elevation-and-zenith-angle}
%\end{figure}

 \begin{figure}[H]
	\centerline{\includegraphics[width=\textwidth]{zenith-and-elevation-angles-schema.jpg}}
	\caption{Angle zénithal et angle d'élévation.}
	\label{fig:elevation-and-zenith-angle}
\end{figure}

L'angle d'élévation $\alpha$ est plus complexe à calculer, car il est fonction de la latitude du point étudié, de \emph{l'angle de déclinaison} et de \emph{l'angle horaire}. On a ainsi, d'après \emph{PVEducation.org} \cite{pos_elevation_angle}, pour $\phi$ la latitude du point étudié :

\[
	\alpha = \sin^{-1} \left(
		\sin \delta \sin \phi +
		\cos \delta \cos \phi \cos \left( \text{angle horaire} \right)
	\right)
\]

L'angle de déclinaison $\delta$ est \emph{``l'angle entre la droite joignant les centres du Soleil et de la Terre et le plan de l'équateur terrestre''} \cite{mouvement_terre}. Il dépend de la position de la Terre autour du Soleil, et donc du \emph{jour de l'année} (allant de 1 à 365).

\[
	\delta = \sin^{-1} \left(
		\sin \left( 23,45\degree \right)
		\sin \left(
			\frac{360}{365}
			\left(
				\text{jour de l'année} - 81
			\right)
		\right)
	\right)
\]



\subsubsection{Angle horaire et temps solaire} % XXX


Pour calculer l'heure solaire à partir de l'heure locale, il faut calculer valeur appelée facteur de correction du temps, qui est la différence entre l'heure solaire et l'heure locale.


\[
	\text{longitude du méridien local} = 15 \degree \cdot \Delta T_{GMT}
\]

\begin{gather*}
	B = \frac{360}{365} \left( d - 81 \right) \\
	\text{équation du temps} = 9,87 \sin \left( 2B \right) - 7,53 \cos \left( B \right) - 1,5 \sin \left( B \right)
\end{gather*}

\begin{multline*}
	\text{facteur de correction de l'heure} = 4 \left( \text{longitude} - \text{longitude du méridien local} \right) \\ + \text{équation du temps}
\end{multline*}

\[
	\text{heure solaire} = \text{heure locale} + \frac{\text{facteur de correction de l'heure}}{60}
\]

\[
	\text{angle horaire} = 15 \degree \left(
		\text{heure solaire} - 12
	\right)
\]

$\Delta T_{GMT}$ est le décalage horaire, en heures, par rapport à l'heure GMT. Ainsi, par exemple, $\Delta T_{GMT}$ vaudra $1$ pour Paris, car son fuseau horaire est GMT+1.

\subsubsection{Combinaison des paramètres}
La combinaison de tous ces paramètres permet de calculer une approximation de l'éclairement énergétique solaire en un point à partir de la position de ce point, de la date, de l'heure locale et du fuseau horaire seulement.
Le fuseau horaire doit être indiqué explicitement car on ne peut le déduire simplement à partir de la latitude, les limites des fuseaux horaires tendant à suivre les frontières des pays au lieu d'être rectilignes, comme le montre la figure~\ref{fig:timezones}.+

\begin{figure}[!ht]
  \centering
      \makebox[\textwidth]{\includegraphics[width=\paperwidth]{timezones.png}}
  \caption{Les fuseaux horaires, dont les limites sont représentées par des lignes rouges, suivent souvent les contours des pays.}
  \label{fig:timezones}
\end{figure}

Cette approximation ne tient pas compte des nuages ou d'autres phénomènes climatiques pouvant altérer le parcours des rayons du Soleil, mais reste assez proche de la réalité par temps clair.

\clearpage
\begin{figure}[!htb]
	\centering
	{ \Large \textbf{Annexe : structure du modèle} \par\medskip }
	\centerline{\includegraphics[width=\paperwidth - 2cm]{graph.png}}
	\caption{Structure du modèle.}
	\label{fig:model-structure}
\end{figure}

\clearpage
\section{Modélisation informatique : écriture du programme}

On dispose à présent d'un petit modèle climatique.
Pour le faire fonctionner sur ordinateur, il va falloir le traduire en un algorithme qu'on intègrera dans un programme informatique.

\subsection{Bases du fonctionnement d'un programme}

Pour être exécuté, un programme doit être traduit en une série d'instructions appelée \emph{code machine}.
Ce code n'étant pas conçu pour être lu ou écrit par des êtres humains, on écrit en pratique les programmes dans des \emph{langages de programmation}, conçus pour être compréhensibles à la fois par le programmeur et par l'ordinateur.

Le texte formant un programme, écrit dans un langage de programmation, est appelé \emph{code source} du programme.
L'ordinateur traduit le code source en code machine à l'aide un programme appelé \emph{compilateur}.
Il est alors libre d'exécuter le code machine.


\subsection{Choix du langage}

Il existe de nombreux langages de programmation, chacun ayant ses propres atouts et points faibles.
Pour ce TPE, j'ai choisi d'écrire mon programme en \textbf{Rust} \footnote{\url{https://www.rust-lang.org/}}.
Il s'agit d'un langage innovant, extrêmement performant, et prévenant de nombreux bugs grâce à un système qui analyse automatiquement le code source afin de vérifier son exactitude avant la traduction en langage machine.
Ce sont ces qualités, ainsi que ma familiarité avec ce langage, que j'utilise depuis janvier 2015, qui ont motivé mon choix.


\subsection{Interface utilisateur}

J'ai décidé de munir mon programme d'une interface utilisateur simple, afin qu'il soit possible de l'utiliser sans connaissances en programmation. Il s'agit d'une interface dite \emph{en ligne de commande}, et non d'une interface graphique traditionnelle, afin préserver la simplicité du code. On l'utilise avec le clavier uniquement, en les paramètres nécessaires l'un après l'autre (figure~\ref{fig:demo}). La date, l'heure et le fuseau horaire sont tous entrés en même temps.

\begin{figure}[!htbp]
  \centering
      \makebox[\textwidth]{\includegraphics[width=15cm]{program-demonstration.png}}
  \caption{Un exemple d'utilisation de mon programme, avec pour paramètres les coordonnées GPS de la tour Eiffel (48.8582° N, 2.2945° E) et la date et l'heure d'écriture de ces lignes. Seuls les paramètres d'entrée (latitude, longitude, date et heure locale) ont été écrits au clavier ; les variables intermédiaires et le résultat ont été calculés et affichés automatiquement.}
  \label{fig:demo}
\end{figure}


\FloatBarrier
\subsection{Code source}

Le code source de mon programme est trop long\footnote{252 lignes au total, dont 176 lignes de code Rust (en excluant les commentaires et les espaces).} pour le copier dans son intégralité, mais je l'ai rendu disponible sur GitHub, un site de développement collaboratif.\footnote{\url{https://github.com/yberreby/rust-climate}}
Vous pouvez en télécharger une copie en cliquant sur "Download ZIP" dans l'interface GitHub.\footnote{\url{https://github.com/yberreby/rust-climate/archive/master.zip}}
Il est également possible de le lire directement depuis votre navigateur, en cliquant sur les noms des fichiers dans l'interface Web.
Le code est dans \texttt{src}.

En voici un extrait (fichier \texttt{src/irradiance.rs}) :

\inputminted[linenos]{rust}{rust-climate/src/irradiance.rs}



\clearpage
\section{Validation expérimentale : essai du programme}

Maintenant qu'on dispose d'un programme fonctionnel, il est nécessaire de le \emph{tester} pour vérifier si les résultats qu'il produit correspondent à la réalité.

\subsection{Résultats attendus}

D'après Wikipédia, l'éclairement énergétique solaire \emph{maximal} au niveau de la mer serait d'environ \SI{1000}{\watt\per\square\meter} \cite{earth_irradiance_wiki}, mais aucune source n'est citée.
D'après une autre source (Green Rhino Energy, une société spécialisée dans l'énergie solaire) \cite{green_rhino_irradiance}, il serait plus proche de \SI{960}{\watt\per\square\meter}.
Cette valeur ne peut être atteinte que quand les rayons du Soleil arrivent perpendiculairement au plan auquel appartient le point étudié.
Ce n'est \emph{pas} le cas tous les jours au midi solaire, contrairement à ce que l'on pourrait penser.
Si c'était le cas, tous les pays seraient, chaque jour de l'année, au midi solaire, soumis à une irradiation solaire équivalente, ce qui est évidemment faux.


\subsection{Résultats obtenus}

Par définition, l'éclairement énergétique maximal est atteint lorsque la masse d'air est égale à $1$.
Cependant, on va en pratique chercher à la faire \emph{tendre} vers cette valeur, car tenter de l'égaler exactement serait inutile en raison de l'imprécision des valeurs mesurées servant de base au calcul de l'éclairement énergétique.

J'ai donc fait fonctionner mon programme avec divers paramètres jusqu'à obtenir une masse d'air tendant vers $1$, et j'obtins un éclairement énergétique maximal de \textbf{\SI{947.07}{\watt\per\square\meter}} au niveau de l'équateur, au Gabon (0\degree N, 13\degree E), à 12h20 heure locale (midi heure solaire), le 22 septembre 2016 (équinoxe d'automne).
Cette valeur est proche des \SI{960}{\watt\per\square\meter} annoncées par Green Rhino Energy : l'essai du programme est donc concluant.

\begin{figure}[!ht]
  \centering
      \makebox[\textwidth]{\includegraphics[width=15cm]{program-maximum.png}}
  \caption{Mon programme affiche un éclairement énergétique solaire maximal de \SI{947.07}{\watt\per\square\meter}, ce qui est proche des 960 à \SI{1000}{\watt\per\square\meter} attendus.}
  \label{fig:maximum}
\end{figure}



\clearpage
\section{Conclusion}

Ainsi, après m'être assuré de comprendre les phénomènes physiques impliqués dans l'irradiation solaire, je les ai intégrés dans un modèle mathématique simple à l'aide des sources citées plus tôt. J'ai ensuite écrit un programme informatique permettant de faire fonctionner ce modèle en réalisant automatiquement tous les calculs nécessaires.

L'éclairement énergétique, que mon modèle permet de calculer, est une donnée qui peut servir de base à une construction plus élaborée.
Ainsi, avec des connaissances et des moyens suffisants, on pourrait imaginer rendre ce modèle plus précis en y ajoutant, par exemple, la prise en compte des nuages, qui absorbent beaucoup d'énergie solaire, ou de la courbure de la Terre\footnote{Le calcul de la masse d'air est en effet une simplification, qui considère le sol comme un plan et non comme la surface d'une sphère, ce qui entraîne de petites imprécisions.}.
On pourrait aussi le rendre capable de prévoir la température, mais il faudrait pour cela qu'il intègre l'effet de serre, les vents, le phénomène de convection atmosphérique, et mille autres données !
On se rend bien vite compte que la complexité augmente de façon exponentielle avec le nombre de facteurs pris en considération.

Cet exemple illustre les défis que doivent relever les chercheurs en sciences du climat : si la modélisation \emph{approximative} d'une seule variable, ignorant nombre de facteurs, nécessite tout de même de nombreux calculs et environ 200 lignes de code, à combien plus forte raison la construction d'un modèle climatique complet, comprenant non seulement l'irradiation solaire mais aussi les vents, la mesure dans laquelle les matériaux à la surface de la Terre absorbent le rayonnement solaire, la formation de nuages, l'humidité de l'air, etc., ne mobilisera-t-elle pas d'incroyable ressources matérielles et humaines !


\clearpage
\begin{thebibliography}{99}

\bibitem{how_do_climate_models_work} 
	Kaitlin Alexander,
	\emph{How do climate models work?}\\
	\url{http://climatesight.org/2012/01/20/how-do-climate-models-work/}

\bibitem{properties_of_sunlight} 
	PVEducation.org,
	\emph{Properties of Sunlight}\\
	\url{http://www.pveducation.org/pvcdrom/properties-of-sunlight/}

\bibitem{pos_air_mass}
	PVEducation.org,
	\emph{Air Mass}\\
	\url{http://www.pveducation.org/pvcdrom/properties-of-sunlight/air-mass}

\bibitem{pos_solar_time}
	PVEducation.org,
	\emph{Solar Time}\\
	\url{http://www.pveducation.org/pvcdrom/properties-of-sunlight/solar-time}

\bibitem{pos_elevation_angle}
	PVEducation.org,
	\emph{Elevation Angle}\\
	\url{http://www.pveducation.org/pvcdrom/properties-of-sunlight/elevation-angle}

\bibitem{pos_declination_angle}
	PVEducation.org,
	\emph{Declination Angle}\\
	\url{http://www.pveducation.org/pvcdrom/properties-of-sunlight/declination-angle}

\bibitem{pos_calculation_of_solar_insolation}
	PVEducation.org,
	\emph{Calculation of Solar Insolation}\\
	\url{http://www.pveducation.org/pvcdrom/properties-of-sunlight/calculation-of-solar-insolation}

\bibitem{mouvement_terre}
	Université du Maine,
	\emph{Mouvement de la Terre}\\
	\url{http://ressources.univ-lemans.fr/AccesLibre/UM/Pedago/physique/02/divers/mouveter.html}

\bibitem{air_mass_wiki}
	Wikipédia,
	\emph{Air Mass (solar energy)}\\
	\url{https://en.wikipedia.org/wiki/Air_mass_(solar_energy)}

\bibitem{equation_temps_wiki}
	Wikipédia,
	\emph{Équation du temps}\\
	\url{https://fr.wikipedia.org/wiki/%C3%89quation_du_temps}
	
\bibitem{irradiation_solaire_wiki}
	Wikipédia,
	\emph{Irradiation solaire}\\
	\url{https://fr.wikipedia.org/wiki/Irradiation_solaire}
	
\bibitem{temps_solaire_wiki}
	Wikipédia,
	\emph{Temps solaire}\\
	\url{https://fr.wikipedia.org/wiki/Temps_solaire}
	
\bibitem{earth_irradiance_wiki}
	Wikipédia,
	\emph{Solar Irradiance - Earth}\\
	\url{https://en.wikipedia.org/wiki/Solar_irradiance#Earth}
	
\bibitem{green_rhino_irradiance}
	Green Rhino Energy,
	\emph{Annual Solar Irradiance, Intermittency and Annual Variations}\\
	\url{http://www.greenrhinoenergy.com/solar/radiation/empiricalevidence.php}

\end{thebibliography}


\end{document}
