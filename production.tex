% Préambule
% ---
\documentclass[12pt]{article}

% Paquets utilisés
% ---
%\usepackage{amsmath} % Advanced math typesetting
\usepackage[utf8]{inputenc} % Unicode support (Umlauts etc.)w
\usepackage[french]{babel} % Change hyphenation rules
\usepackage[T1]{fontenc} % for farenheit
\usepackage{textcomp}
\usepackage{gensymb}
\usepackage{hyperref}
\usepackage{minted}
%\usepackage{graphicx} % Add pictures to your document
%\usepackage{listings} % Source code formatting and highlighting

\begin{document}

%%
% Page de couverture sur-mesure
%%

\begin{titlepage}
	\centering
	
	{\scshape\large TPE Mathématiques et SVT\par}
	\vspace{0.2cm}	
	{ \Large Thème : l'aléatoire, l'insolite, le prévisible\par }
	Axe de recherche : comprendre le présent, penser le futur\par
	\vspace{1.5cm}

	{\scshape\LARGE La modélisation informatique du climat \par}
	\vspace{1cm}
	{\huge\bfseries Comment peut-on modéliser le rayonnement solaire, et traduire cette modélisation mathématique en programme informatique ?\par}

	\vspace{1cm}
	{\Large\itshape Yohaï Eliel Berreby\par}
	
	\vfill
	
	
	{\Large\bfseries Établissement : CNED\par}
	\vspace{0.2cm}
	{\Large \bfseries Série : S }

	\vfill

	{\large \today\par}
\end{titlepage}


\clearpage
\tableofcontents{}
\clearpage

%%
% Début du document
%%

\section{Introduction} 

\subsection{Des enjeux actuels} % des enjeux actuels

Qui n'a jamais eu besoin d'allumer sa télévision ou de consulter son téléphone pour savoir quel temps il ferait avant de prendre un avion, de partir en vacances ou tout simplement d'aller se promener en forêt ?
Les prévisions météorologiques, toutes communes et faciles d'accès qu'elles soient, n'en présentent cependant pas moins des enjeux importants.
Ces enjeux sont \textbf{économiques}, avec par exemple la météo agricole, un outil précieux pour des millions d'agriculteurs à travers le monde, mais aussi \textbf{humains}, car prévoir une crue ou une tempête, ne serait-ce que quelques heures à l'avance, peut permettre de sauver des vies en évacuant les populations en danger.

Ces prévisions sont aujourd'hui littéralement à \textit{portée de main} avec l'avènement des applications mobiles, et pourtant elles reposent sur des constructions d'une grande complexité à la croisée des mathématiques, de la physique et de l'informatique : \textbf{les modèles}.

\subsection{Rapport au thème}

La modélisation climatique s'inscrit naturellement dans le thème choisi, à savoir \textit{"l'aléatoire, l'insolite, le prévisible"}, car, outre l'exploration du passé qu'elle permet -- un point sur lequel on reviendra plus loin --, l'un de ses principaux intérêts est la réalisation de \textbf{prévisions météorologiques}.

Elle est également aujourd'hui indissociable de l'informatique, discipline qui passionne votre serviteur, d'où le choix du sujet \textit{"la modélisation informatique du climat"}.
En effet, le volume de données que doit traiter tout modèle climatique à partir d'un certain degré de complexité est bien trop important pour qu'un être humain, même muni d'une calculatrice, puisse réaliser les calculs nécessaires : on a donc recours à des ordinateurs pour cela, car ils sont capables de réaliser ces calculs à des vitesses dépassant de très loin les capacités humaines.

\subsection{Démarche expérimentale}

Le rôle primordial des modèles dans un domaine aussi important que la météorologie m'a conduit à vouloir \textbf{explorer leur fonctionnement}.
Pour ce faire, j'ai choisi de créer un modèle simple, et de le traduire en un programme informatique, avec comme variable l'intensité du rayonnement solaire en un point de la Terre.

Cet exemple, d'une grande simplicité en comparaison avec les modèles qu'utilisent des organismes comme Météo France, présente cependant de multiples avantages :

\begin{itemize}
  \item sa \textbf{simplicité} le rend réalisable par un élève seul dans le cadre d'un TPE
  \item les calculs qu'il implique restent cependant suffisamment complexes pour \textbf{justifier la création d'un programme informatique} les réalisant automatiquement
  \item le rayonnement solaire est \textbf{relativement simple à modéliser}, car plus prévisible que le vent ou les intempéries
\end{itemize}


J'ai donc abouti, à l'aide de ressources trouvées sur Internet, citées plus loin, à une série de calculs permettant de calculer \textbf{l'éclairement énergétique solaire} (ou \textit{irradiance solaire}) d'un point en fonction de sa position sur Terre, de la date et de l'heure.
J'ai ensuite intégré ces calculs dans un programme, et procédé à la validation expérimentale de celui-ci en le faisant fonctionner et en comparant les résultats obtenus à ceux attendus.

\clearpage
\section{Construction de notre modèle}

\subsection{Qu'est-ce qu'un modèle ?}

On parle de modèles mathématiques, scientifiques, climatiques, ou encore informatiques -- mais qu'est-ce au juste qu'un modèle ?
Ces notions, bien que diverses, se rejoignent en un point : il s'agit de représentations de phénomènes, suffisamment simples pour qu'on puisse les appréhender et les manipuler, mais qui restent les plus proches possibles de la réalité, sans quoi elles seraient inutiles.
En ce sens, l'on peut dire que tout modèle est une \textbf{approximation du réel}.

Le modèle \textit{mathématique} est une construction formelle qui bien souvent sert de base au modèle informatique. Les lois qui régissent le premier servent de base au second, qui, soumis à des contraintes physiques (vitesse de calcul, de transfert des données, capacité de stockage d'un volume de données grandissant), en est la traduction imparfaite, empirique, gouverné par l'expérience, approximative. Il dépend du matériel avec lequel on l'utilise, et évolue au fur et à mesure que celui-ci devient plus rapide et capable de traiter de grands volumes de données.



[à faire - ici on parle du point de vue mathématique]

% \section{Démarche théorique : choix du paramètre de l'éclairement énergétique}

\clearpage
\section{Application : écriture de notre programme}

\subsection{Bases du fonctionnement d'un programme}

Pour être exécuté, un programme doit être traduit en une série d'instructions appelée \textit{code machine}. Ce code n'étant pas conçu pour être lu ou écrit par des êtres humains, on écrit en pratique les programmes dans des \textit{langages de programmation}, conçus pour être compréhensibles à la fois par le programmeur et par l'ordinateur.

Le texte formant un programme, écrit dans un langage de programmation, est appelé \textit{code source} du programme. L'ordinateur traduit le code source en code machine à l'aide un programme appelé \textit{compilateur}. Il est alors libre d'exécuter le code machine.

\subsection{Choix du langage}
Il existe de nombreux langages de programmation, chacun ayant ses propres atouts et points faibles. Pour ce TPE, j'ai choisi d'écrire mon programme en \textbf{Rust} (\url{https://www.rust-lang.org/}). Il s'agit d'un langage innovant, extrêmement performant, et prévenant de nombreux bugs grâce à un système qui analyse automatiquement le code source afin de vérifier son exactitude avant la traduction en langage machine. Ce sont ces qualités, ainsi que ma familiarité avec le langage, que j'utilise depuis janvier 2015, qui ont motivé mon choix.

[à étoffer]

\subsection{Code source du programme final}
Le code source de mon programme est trop long pour le copier dans son intégralité, mais il est disponible publiquement sur GitHub, un site de développement collaboratif, à l'adresse suivante : \url{https://github.com/yberreby/rust-climate}. Vous pouvez en télécharger une copie sur votre disque dur en cliquant sur "Download ZIP" dans l'interface GitHub, ou sur le lien suivant : \url{https://github.com/yberreby/rust-climate/archive/master.zip}.

Il est également possible de le lire directement depuis votre navigateur, en cliquant sur les noms de fichiers. Le code est dans \texttt{src}.

En voici un extrait :

\inputminted[linenos]{rust}{temperature.rs}

[à étoffer]
[le code est tronqué car certaines lignes sont trop longues

\clearpage
\section{Validation expérimentale : essai du programme}

[à faire]

\clearpage
\section{Conclusion}

[à faire]


\begin{thebibliography}{9}

\bibitem{how_do_climate_models_work} 
Kaitlin Alexander, \textit{How do climate models work?}

\url{http://climatesight.org/2012/01/20/how-do-climate-models-work/}


\bibitem{properties_of_sunlight} 
PVEducation.org, \textit{Properties of Sunlight}

\url{http://www.pveducation.org/pvcdrom/properties-of-sunlight/}


\bibitem{cahier_globe} 
GLOBE-SWISS, \textit{Climat, météo et atmosphère}

\url{http://www.globe-swiss.ch/files/Downloads/779/Download/Cahier_Atmosphere.pdf}


\bibitem{qu_est_ce_qu_un_modele}
Jean-Louis Le Moigne, \textit{Qu'est-ce qu'un modèle ?}

\url{http://archive.mcxapc.org/docs/ateliers/lemoign2.pdf}


\end{thebibliography}


\end{document}
